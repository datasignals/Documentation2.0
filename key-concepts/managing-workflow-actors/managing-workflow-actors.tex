\documentclass{tufte-handout}

%\geometry{showframe}% for debugging purposes -- displays the margins

\usepackage{amsmath}

% Set up the images/graphics package
\usepackage{graphicx}
\setkeys{Gin}{width=\linewidth,totalheight=\textheight,keepaspectratio}
\graphicspath{{graphics/}}

\title{Managing Workflow Actors: Contract Management\thanks{Facilitated by Polkadot Parachains and NFT Marketplaces} \\
\large Management of Actors participating in a supply chain workflow.}
\author[Gary Mawdsley]{Gary Mawdsley CTO/CEO Lockular Limited}
\date{February 2024\thanks{Updated April 2024 to consider Medicines supply chain context}}    % if the \date{} command is left out, the current date will be used

% The following package makes prettier tables.  We're all about the communication!
\usepackage{booktabs}

% The units package provides nice, non-stacked fractions and better spacing
% for units.
\usepackage{units}


% The fancyvrb package lets us customize the formatting of verbatim
% environments.  We use a slightly smaller font.
\usepackage{fancyvrb}
\fvset{fontsize=\normalsize}

% Small sections of multiple columns
\usepackage{multicol}

% Provides paragraphs of dummy text
\usepackage{lipsum}
\usepackage{tabularx}
\usepackage{booktabs}

% These commands are used to pretty-print LaTeX commands
\newcommand{\doccmd}[1]{\texttt{\textbackslash#1}}% command name -- adds backslash automatically
\newcommand{\docopt}[1]{\ensuremath{\langle}\textrm{\textit{#1}}\ensuremath{\rangle}}% optional command argument
\newcommand{\docarg}[1]{\textrm{\textit{#1}}}% (required) command argument
\newenvironment{docspec}{\begin{quote}\noindent}{\end{quote}}% command specification environment
\newcommand{\docenv}[1]{\textsf{#1}}% environment name
\newcommand{\docpkg}[1]{\texttt{#1}}% package name
\newcommand{\doccls}[1]{\texttt{#1}}% document class name
\newcommand{\docclsopt}[1]{\texttt{#1}}% document class option name

\usepackage{titlesec} % Make sure to include this package

% Custom appendix command
\newcommand{\customappendix}{
    \clearpage % Ensure a page break
    \appendix % Mark the beginning of appendices
    \pagestyle{empty} % Optional: Removes header/footer if desired
    \titleformat{\section}[display]{\normalfont\Large\bfseries}{\appendixname~\thesection}{0pt}{\Large}
    \titlespacing*{\section}{0pt}{50pt}{40pt}
    \section*{Appendices} % This adds the title "Appendices"
    \addcontentsline{toc}{section}{Appendices} % Optional: Adds "Appendices" to the table of contents
}

\begin{document}

\maketitle% this prints the handout title, author, and date

\begin{abstract}
\noindent This paper outlines how actors involved in an NFT Marketplace based supply chain are organised and managed under
a strict provenance regime. In particular it talks of domain specific smart contracts, multi-sig and how relationships are
tracked and recorded.

\end{abstract}

%\printclassoptions
\marginnote{See the paper on coupling real and virtual worlds.}

\section{Supply Chain tracking}\label{sec:page-layout}
Supply chain tracking is important across many domains. These include materials procurement in the defence sector,
distribution of public examinations, assembly and distribition of drugs from manifacture, to wholssale, hospitals
and patients to name a few.

It is of primary importance to track relationships of organisations and the individuals that participate (the stakeholders) in order for the
provenance record is complete. For example the individual batching drugs and the patient to whom the medicines are dispensed.
Other examples include recording organisational and team  structures for silicon chip design projects allowing design contributions
to be tracked at the team and individual level.

Lockular's WAM system is a smart contract based app that is used to record organisational structures, team members and organisational
relationships across the group of stakeholders which in all likelihood span organisations. WAM is foundational in establishing multi-sig
across the group of stakeholders.

\section{Workflow Actor Management System}\label{sec:page-layout}
Workflow Actor Management System (WAM) is a smart contract-based app used to facilitate multi-sig for organisational structures. The actual structures
represented depend on the supply chain domain and are determined by the smart contract definitions. Often they involve globally labelled divisions divisions,
teams, and members. Use of multi-sig across teams significantly enhances security, authorisation, and decision-making processes. WAM requires configuration for a
specific domain. Configuration is made by a smart contract definition that addresses key considerations in respect of the domain:

\begin{itemize}
    \item Representation of the stakeholder structure. In all likelihood this addresses inter-organisational hierarchical structures:
    \begin{itemize}
        \item Organisations are often divided by division which often have geographical connotation
        \item Stakeholders work in a division and are assigned one or many roles
        \item Role and division may have restricted approval and, or access
    \end{itemize}
    \item Multi-Sig Rules Definition: Rules are defined for multi-sig within the organisational structures. This involves deciding how many
    signatures (approvals) are required for different types of transactions or decisions at each level of the those organisations involved (division, team,
    member). This requires making decisions around the following:
    \begin{itemize}
        \item Global Rules: These apply to the entire organisation, for example financial thresholds above which multi-sig is required.
        \item Divisional/Team Rules: Specific rules for divisions or teams, based on the sensitivity or importance of the decisions.
        \item Role-Based Rules: Rules that apply based on the role of a member within the organization, such as managers having different thresholds
        or permissions.
    \end{itemize}
    \item Adjustment of the smart contract to the Multi-Sig Mechanism: This involves creating a contract structure that:
    \begin{itemize}
        \item Store Proposals: Create a structure to store proposals or transactions that require approval. Each proposal should include details
        like the type of decision, the required number of signatures, and the current number of approvals.
        \item Manage Signatures: Develop functions that allow eligible members to sign or approve proposals. Ensure that each member can only sign
        a proposal once and that their signature is authenticated.
        \item Execute Decisions: Once a proposal has received the required number of signatures, execute the decision. This could involve changing
        the state of the smart contract, transferring assets, or any other action that was proposed.
    \end{itemize}
    \item Role-Based Access Control (RBAC) considerations. Integrate RBAC to manage who can create proposals and who can sign them. This involves:
    \begin{itemize}
        \item Defining Roles: Define roles within your organization, such as admin, manager, team leader, etc.
        \item Assigning Permissions: Assign permissions to these roles regarding who can initiate proposals and who is authorised to sign them.
        \item Role Assignment: Assign roles to individual members within your organisational structure.
    \end{itemize}
    \item User Interface for Proposals and Signatures. Develop a user-friendly interface for the domain that allows members to:
    \begin{itemize}
        \item Join the workflow at the appropriate points signing in via their wallets. Generically, the tasks they address are called proposals.
        Here stakeholders are able to view open proposals that they are eligible to sign.
        \item Sign Proposals: A simple mechanism for stakeholders to sign or approve proposals they agree with.
        \item Track Progress: Stakeholders are able to see the current number of signatures on a proposal and whether it has met the required
        threshold for approval.
        \item Wallet Integration: For a stakeholder to participate in a multi-sig scheme within the Polkadot ecosystem, their Polkadot.js wallet
        is a key interface. The wallet stores the stakeholder's private keys securely and is used to sign transactions or approvals as
        part of the multi-sig process.
    \end{itemize}
    \item Multi-Sig Wallet integration (polkadot.js)
    \begin{itemize}
        \item Wallet Integration: For a stakeholder to participate in a multi-sig scheme within the Polkadot ecosystem, their Polkadot.js wallet
        is one of the key interfaces. This wallet stores their private keys securely and be used to sign transactions or approvals as part of the multi-sig
        process.
        \item Signing Transactions: In a multi-sig setup, a transaction (such as approving a supply chain action or executing a contract function)
        requires signatures from multiple stakeholders before it can be executed on the blockchain. Stakeholders use their Polkadot.js wallets to
        review and sign these transactions, with the wallet ensuring the security of their private keys during the signing process.
        \item Smart Contract Interaction: When interacting with a smart contract that requires multi-sig approvals, stakeholders would use their
        Polkadot.js wallets to initiate transactions or to sign off on proposals made by others. The smart contract would then verify that the required
        number of signatures has been collected before executing the agreed-upon action.
        \item Configuration and Management: The initial setup of a multi-sig arrangement, including specifying which accounts are part of the multi-sig and
        defining the rules (e.g., how many signatures are needed for different types of transactions), is managed through interactions initiated
        from Polkadot.js wallets. This setup should be encoded in the domain specific smart contract governing the multi-sig process.
        \item Security and Authentication: The Polkadot.js wallet plays a crucial role in the security and authentication process. It ensures that
        only authorised stakeholders, who hold the corresponding private keys, can sign off on multi-sig transactions. This adds an additional layer
        of security to the multi-sig process, as the wallet's security features protect against unauthorised access.
    \end{itemize}
\end{itemize}

\section{Benefits}\label{sec:page-layout}
Once Lockular WAM is configured for a given domain this operates as a multi-sig smart contract based system [within as Polkadot Parachain and Marketplace]
and thereby manages supply chain activities across participating stakeholders, spanning multiple organisations, several key significant pieces of information
can be discovered and tracked. Transparency and traceability are among the core benefits of using blockchain technology in supply chain management. In detail
here's what can be discovered:
\begin{itemize}
\item Transaction History and Provenance
\begin{itemize}
    \item Asset Origin: The origin of goods can be traced back to their source, including details about the manufacturer, the location, and the date of
    production. This is particularly valuable for verifying the authenticity of products and ensuring they meet regulatory standards.
    \item Ownership Transfers: Every transfer of goods between stakeholders in the supply chain is recorded on the blockchain, providing a clear history
    of ownership. This helps in verifying the chain of custody and ensuring that goods have not been tampered with or diverted.
\end{itemize}
\item Compliance and Certification
\begin{itemize}
    \item Regulatory Compliance: Information about compliance with relevant regulations and standards can be stored on the blockchain. This includes
    certifications, inspection reports, and compliance checks, which are crucial for industries with strict regulatory requirements, such as
    pharmaceuticals and food.
    \item Certification Verification: Certifications for organic, fair-trade, or sustainability credentials can be verified by accessing the blockchain. This
    supports claims about product quality and ethical standards, enhancing consumer trust.
\end{itemize}
\item Operational Insights
\begin{itemize}
    \item Supply Chain Efficiency: By analysing transaction times and movement of goods, organisations can identify bottlenecks or inefficiencies in the
    supply chain. This data can inform operational improvements and optimizations.
    \item Inventory Management: Real-time data on the movement of goods provides insights into inventory levels across the supply chain, helping in
    demand forecasting and reducing the risk of overstocking or stockouts.
\end{itemize}
\item Security and Risk Management
\begin{itemize}
    \item Fraud Prevention: The immutable nature of blockchain records helps in preventing fraud, as any attempt to alter transaction history or forge
    documents would be evident.
    \item Dispute Resolution: The clear and auditable trail of transactions and approvals facilitates faster resolution of disputes between stakeholders,
    as there is a reliable record of agreements and transactions.
\end{itemize}
\item Stakeholder Collaboration and Trust
\begin{itemize}
    \item Transparent Operations: The visibility of supply chain activities to all authorised stakeholders fosters a culture of transparency, building
    trust among participants.
    \item Collaboration Opportunities: Shared insights into the supply chain can reveal opportunities for collaboration, such as joint efforts to
    improve sustainability or efficiency.
\end{itemize}
\item Consumer Insights
\begin{itemize}
    \item Product Journey: End consumers can access information about the journey of a product from production to retail, enhancing transparency and
    potentially influencing purchasing decisions.
    \item Authenticity Verification: Consumers concerned about counterfeit products can verify the authenticity of their purchases through the blockchain,
    enhancing brand trust.
\end{itemize}
\item Errata Management significantly improving the way errors, defects, or recalls are handled across the supply chain.
\begin{itemize}
    \item Real-Time Tracking: With every product and batch tracked on the blockchain, identifying where and when an issue arises becomes much more
    straightforward. This could include defects in manufacturing, issues during transportation, or any other problems that occur along the supply chain.
    \item Immutable Records: The blockchain's immutable record ensures that once an issue is logged, it cannot be altered or deleted, providing a transparent
    and accurate history of product quality and handling.
    \item Smart Contract Triggers: Smart contracts can be programmed to automatically notify relevant stakeholders when an issue is detected. For example,
    if a product fails to meet certain quality standards at a checkpoint, the smart contract could immediately alert the manufacturer, distributor, and
    even end consumers if necessary.
    \item Targeted Recalls: By having detailed records of the entire supply chain, recalls can be conducted more efficiently and precisely. Instead of
    recalling vast quantities of products, companies can target only those batches or items that are affected.
    \item Consumer Safety: For industries like pharmaceuticals and food, rapid response to quality issues is critical for consumer safety. Blockchain can
    facilitate this by quickly tracing and removing compromised products from the supply chain.
\end{itemize}
\end{itemize}

\section{Summary}\label{sec:page-layout}
The implementation of a multi-sig smart contract system for supply chain management on the blockchain offers comprehensive visibility into supply chain
activities. It not only enhances operational efficiency, compliance, and security but also builds a foundation of trust among stakeholders and with end
consumers. This level of transparency and traceability is transformative, potentially reshaping how supply chains are managed across industries.

\bibliographystyle{plain}
\bibliography{thebib} % This assumes you have a file named 'your_bib_file.bib'

\end{document}
